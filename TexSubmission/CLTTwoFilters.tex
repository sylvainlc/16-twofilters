\label{sec:CLTTwoFilters}
We now establish CLT for the two-filter algorithms. Note first that under assumptions A\ref{assum:bound-likelihood}, A\ref{assum:borne-FFBS} and A\ref{assum:borne-TwoFilters}, for all $0\le s,t\le T$ a CLT may be derived for the weighted samples $\{(\epart{s}{\ell},\ewght{s}{\ell})\}_{\ell=1}^N$ and $\{ (\ebackpart{t}{T}{i},\ebackwght{t}{T}{i}) \}_{i=1}^N$ which target respectively the filtering distribution $\filt{\Xinit,s}$ and the backward information filter $\backDist[][\gamma]{t}{T}$. By Propositions~\ref{prop:CLT-forward} and~\ref{prop:CLT-backward}, there exist $\asymVar[\Xinit]{s}{}$ and $\backasymVar[\gamma]{t}{T}{}$ such that for any $h \in \functionset[b]{X}$,
\begin{align}
N^{1/2} \sum_{i=1}^N \frac{\ewght{s}{i}}{\sumwght{s}} \left(h(\epart{s}{i}) - \filt{\Xinit,s}[h]\right) &\dlim_{N \to \infty} \mathcal{N}\left(0,\asymVar[\Xinit]{s}{h - \filt{\Xinit,s}[h]}\right)\eqsp,\label{eq:CLT:forward}\\
N^{1/2} \sum_{j=1}^N \frac{\ebackwght{t}{T}{j}}{\backsumwght{t}{T}} \left(h(\ebackpart{t}{T}{j})- \backDist[][\gamma]{t}{T}[h]\right) &\dlim_{N \to \infty} \mathcal{N}\left(0,\backasymVar[\gamma]{t}{T}{h-\backDist[][\gamma]{t}{T}[h]}\right)\label{eq:CLT:backward}\eqsp.
\end{align}
Theorem~\ref{thm:CLT-Two-Populations} establishes a CLT for the weighted sample $\{ \swght{s}{i}\ebackwght{t}{T}{j}, (\epartpred{s}{i},\ebackpart{t}{T}{j})\}_{i,j=1}^N$ which targets the product distribution $\filt{\Xinit,s}\otimes \backDist[][\gamma]{t}{T}$. As an important consequence, the asymptotic variance of the weighted sample $\{\swght{s}{i}\ebackwght{t}{T}{j}, (\epartpred{s}{i},\ebackpart{t}{T}{j})\}_{i,j=1}^N$ is the sum of two contributions, the first one involves $\asymVar[\Xinit]{s}{}$ and the second one $\backasymVar[\gamma]{t}{T}{}$. Intuitively, this may be explained by the fact that the estimator $\filt{\Xinit,s}^N\otimes\backDist[][\gamma]{t}{T}^N[h]$ is obtained by mixing two independent weighted samples which suggests the following decomposition:
\begin{multline*}
\sqrt{N}\sum_{i,j=1}^N \frac{\swght{s}{i}}{\sumwght{s}}\frac{\ebackwght{t}{T}{j}}{\backsumwght{t}{T}} \tilde{h}_{s,t}(\epartpred{s}{i},\ebackpart{t}{T}{j}) = \sqrt{N}\sum_{j=1}^N \frac{\ebackwght{t}{T}{j}}{\backsumwght{t}{T}} \filt{\Xinit,s}[\tilde{h}_{s,t}(\cdot,\ebackpart{t}{T}{j})]\\
+ \sqrt{N}\sum_{i=1}^N \frac{\swght{s}{i}}{\sumwght{s}} \backDist[][\gamma]{t}{T}[\tilde{h}_{s,t}(\epartpred{s}{i},\cdot)] + \mathcal{E}^{N}_{s,T|t}(\tilde{h}_{s,t}) \eqsp,
\end{multline*}
where $\tilde{h}_{s,t} = h - \filt{\Xinit,s}\otimes\backDist[][\gamma]{t}{T}[h]$ and
\[
\mathcal{E}^{N}_{s,T|t}(h) \eqdef \sqrt{N}\sum_{i,j=1}^N \frac{\swght{s}{i}}{\sumwght{s}}\frac{\ebackwght{t}{T}{j}}{\backsumwght{t}{T}}\left\{h(\epartpred{s}{i},\ebackpart{t}{T}{j}) -  \filt{\Xinit,s}[h(\cdot,\ebackpart{t}{T}{j})]  - \backDist[][\gamma]{t}{T}[h(\epartpred{s}{i},\cdot)]\right\}\eqsp.
\]
A CLT for the two independent first terms is obtained by \eqref{eq:CLT:forward} and \eqref{eq:CLT:backward}. It remains then to prove that $\mathcal{E}^{N}_{s,T|t}(h)$ converges in probability to $0$. However, this cannot be obtained directly from the exponential deviation inequality derived in Theorem~\ref{thm:Hoeffding-Two-Populations} and requires sharper controls of the smoothing error (for instance nonasymptotic $\mathrm{L}_p$-mean error bounds).
Theorem~\ref{thm:CLT-Two-Populations}  provides a direct proof following the asymptotic theory of weighted system of particles developed in \cite{douc:moulines:2008}.
\begin{thm}
\label{thm:CLT-Two-Populations}
Assume that A\ref{assum:bound-likelihood}, A\ref{assum:borne-FFBS} and A\ref{assum:borne-TwoFilters} hold for some $T<\infty$. Then,  for all $0 \leq s<t \leq T$ and all $h \in \functionset[b]{X\times X}$,
\begin{multline*}
\sqrt{N}\left(\sum_{i,j=1}^N \frac{\swght{s}{i}}{\sumwght{s}}\frac{\ebackwght{t}{T}{j}}{\backsumwght{t}{T}} h(\epartpred{s}{i},\ebackpart{t}{T}{j}) - \filt{\Xinit,s}\otimes \backDist[][\gamma]{t}{T}[h]
\right) \\
\dlim_{N \to \infty} \mathcal{N}\left(0,\asymVarJoint{s,t}{T}{h- \filt{\Xinit,s}\otimes \backDist[][\gamma]{t}{T}[h]}\right) \eqsp,
\end{multline*}
where $\asymVarJoint{s,t}{T}{h}$ is defined by:
\begin{equation}
\label{eq:asymVarjoint}
\asymVarJoint{s,t}{T}{h} \eqdef \asymVar[\Xinit]{s}{\int \backDist[][\gamma]{t}{T}(\rmd x_t)h(\cdot,x_t)} +  \backasymVar[\gamma]{t}{T}{\int \filt{\Xinit,s}(\rmd x_{s}) h(x_s,\cdot)}\eqsp,
\end{equation}
with $\asymVar[\Xinit]{s}{}$ and $\backasymVar[\gamma]{t}{T}{}$ are given in Proposition~\ref{prop:CLT-forward} and Proposition~\ref{prop:CLT-backward}.
\end{thm}

\begin{proof}
The proof is postponed to Section~\ref{proof:thm:CLT-Two-Populations}.
\end{proof}
Define
\begin{align*}
\sigma_s &\eqdef \filt{\Xinit,s-1} \otimes \backDist[][\gamma]{s+1}{T} \left[ \int  q^{[2]}(\cdot,x) g_s(x) \rmd x \odot \fakeprior_{s+1}^{-1} \right] \eqsp, \\
\Sigma_s[h] &\eqdef \asymVarJoint{s-1,s+1}{T}{\int q^{[2]}(\cdot;x) g_s(x)  h(x) \rmd x \odot \fakeprior_{s+1}^{-1}} \eqsp.
\end{align*}
Theorem~\ref{thm:CLT-Fearnhead-smoother} provides a CLT for the $\fwt$ algorithm of \cite{fearnhead:wyncoll:tawn:2010}
\begin{thm}[CLT for the $\fwt$ algorithm of \cite{fearnhead:wyncoll:tawn:2010}]
\label{thm:CLT-Fearnhead-smoother}
Assume that A\ref{assum:bound-likelihood}, A\ref{assum:borne-FFBS} and A\ref{assum:borne-TwoFilters} hold for some $T<\infty$. Then,  for all $1\le s \le T-1$ and all $h \in \functionset[b]{X}$,
\begin{equation*}
\sqrt{N}\left( \sum_{i=1}^N \frac{\smwght{s}{T}{i}}{\tilde{\Omega}_{s|T}} h(\smpart{s}{T}{i}) - \post[\Xinit]{s}{T}[h]
\right)
\dlim_{N \to \infty} \mathcal{N}\left(0,\asymVarFearnhead[\Xinit]{s}{T}{h - \post[\Xinit]{s}{T}[h]}\right) \eqsp.
\end{equation*}
where
\begin{multline}
\label{eqdef:asymvarfearnhead} 
\asymVarFearnhead[\Xinit]{s}{T}{h}= \sigma_s^{-2} \left\{ \Sigma_s[h] + \filt{\Xinit,s-1} \otimes \backDist[][\gamma]{s+1}{T} \left[\adjfunc[smooth]{s}{T}{} \odot \fakeprior^{-1}_{s+1} \right] \phantom{\int} \right. \\  \times \left.  \filt{\Xinit,s-1} \otimes \backDist[][\gamma]{s+1}{T}\left[ \int \smwghtfunc{s}{T}(\cdot;x) q^{[2]}(\cdot,x) g_s(x) h^2(x) \rmd x \odot \fakeprior^{-1}_{s+1} \right]\right\} \eqsp.
\end{multline}
\end{thm}

\begin{proof}
The proof is postponed to Section~\ref{proof:thm:CLT-Fearnhead-smoother}.
\end{proof}
For any $h \in \functionset[b]{X}$, define:
\begin{equation}
\label{eq:barh}
\fup{h}{s}(x,x')\eqdef \fakeprior^{-1}_{s+1}(x') h(x) q(x,x')\quad\mbox{and}\quad\fdown{h}{s}(x,x')\eqdef \fakeprior_s^{-1}(x') q(x,x') h(x')\eqsp.
\end{equation}
It follows from the definition of the forward and backward smoothing weights \eqref{eq:definition-smoothing-weight-forward} and \eqref{eq:definition-smoothing-weight-backward} that,
\begin{align}
\label{eq:forward-approximation}
& \sum_{i=1}^N \frac{\smwght{s}{T}{i, \mathrm{f}}}{\tilde{\Omega}^{\mathrm{f}}_{s|T}} h(\epart{s}{i}) =
\frac{\sumwght{s}^{-1} \backsumwght{s+1}{T}^{-1} \sum_{i,j=1}^N \ewght{s}{i} \ebackwght{s+1}{T}{j} \fup{h}{s}(\epart{s}{i},\ebackpart{s+1}{T}{j}) }
{\sumwght{s}^{-1} \backsumwght{s+1}{T}^{-1} \sum_{i,j=1}^N \ewght{s}{i} \ebackwght{s+1}{T}{j}  \fup{\1}{s}(\epart{s}{i},\ebackpart{s+1}{T}{j}) } \eqsp, \\
\label{eq:backward-approximation}
& \sum_{i=1}^N \frac{\smwght{s}{T}{i, \mathrm{b}}}{\tilde{\Omega}^{\mathrm{b}}_{s|T}} h(\ebackpart{s}{T}{i})= \frac{\sumwght{s-1}^{-1} \backsumwght{s}{T}^{-1} \sum_{i,j=1}^N \ewght{s-1}{i} \ebackwght{s}{T}{j} \fdown{h}{s}(\epart{s-1}{i},\ebackpart{s}{T}{j})}{\sumwght{s-1}^{-1} \backsumwght{s}{T}^{-1} \sum_{i,j=1}^N \ewght{s-1}{i} \ebackwght{s}{T}{j}\fdown{\1}{s}(\epart{s-1}{i},\ebackpart{s}{T}{j})} \eqsp,
 \end{align}
 where $\1$ is the constant function which equals 1 for all $x\in\Xset$. These decompositions together with Theorem \ref{thm:CLT-Two-Populations} allow to prove a CLT form the forward and the backward approximations of the marginal smoothing distribution. Theorem~\ref{thm:CLT-Doucet-Forward-Backward-Smoother} is a direct consequence of Proposition~\ref{prop:CLT-backward}, Theorem~\ref{thm:CLT-Two-Populations} and Slutsky Lemma.
\begin{thm}[CLT for the $\bdm$ algorithm of  \cite{briers:doucet:maskell:2010}]
\label{thm:CLT-Doucet-Forward-Backward-Smoother}
Assume that A\ref{assum:bound-likelihood}, A\ref{assum:borne-FFBS} and A\ref{assum:borne-TwoFilters} hold for some $T<\infty$. Then, for all $1 \leq s \leq T-1$ and all $h \in \functionset[b]{X}$,
\begin{equation*}
\sqrt{N}\left( \sum_{i=1}^N \frac{\smwght{s}{T}{i, \mathrm{f}}}{\tilde{\Omega}^{\mathrm{f}}_{s|T}} h(\epart{s}{i}) - \post[\Xinit]{s}{T}[h]
\right) \dlim_{N \to \infty} \mathcal{N}\left(0,\asymVarDoucet[\Xinit]{\forward}{s}{T}{h- \post[\Xinit]{s}{T}[h]} \right)\eqsp,
\end{equation*}
where
\begin{align*}
\asymVarDoucet[\Xinit]{\forward}{s}{T}{h} &\eqdef \asymVarJoint{s,s+1}{T}{H^{\forward}_s} /\{ \filt{\Xinit,s} \otimes \backDist[][\gamma]{s+1}{T}[q \odot \fakeprior_{s+1}^{-1}] \}^2\eqsp,\\
H^{\forward}_s (x,x') &\eqdef h(x) q(x,x') \fakeprior_{s+1}^{-1}(x')\eqsp.
\end{align*}
Similarly,
\begin{equation*}
\sqrt{N}\left( \sum_{i=1}^N \frac{\smwght{s}{T}{i, \mathrm{b}}}{\tilde{\Omega}^{\mathrm{b}}_{s|T}} h(\epart{s}{i}) - \post[\Xinit]{s}{T}[h]
\right)
\dlim_{N \to \infty} \mathcal{N}\left(0, \asymVarDoucet[\Xinit]{\backward}{s}{T}{h- \post[\Xinit]{s}{T}[h]}\right)\eqsp,
\end{equation*}
where
\begin{align*}
\asymVarDoucet[\Xinit]{\backward}{s}{T}{h} &\eqdef \asymVarJoint{s-1,s}{T}{H_s^{\backward} }/\{ \filt{\Xinit,s-1} \otimes \backDist[][\gamma]{s}{T}[q \odot \fakeprior_s^{-1}] \}^2\eqsp,\\
H_s^{\backward} (x,x') &\eqdef  q(x,x') \fakeprior_s^{-1}(x')h(x')\eqsp.
\end{align*}
\end{thm}
Note that $\sigma_s$ and $\Sigma_s[h]$ may be written as:
\[
\sigma_s = \filt{\Xinit,s} \otimes \backDist[][\gamma]{s+1}{T} \left[q\odot \fakeprior_{s+1}^{-1} \right]\times \filt{\Xinit,s-1}\left[ \int  q(\cdot,x) g_s(x) \rmd x\right]
\]
and by Theorem~\ref{thm:CLT-Two-Populations},
\begin{multline*}
\Sigma_s[h] = \asymVar[\Xinit]{s-1}{\int q(\cdot,x)g_s(x)h^1_{s+1}(x)\rmd x} \\
+ \filt{\Xinit,s-1}^2\left[ \int  q(\cdot,x) g_s(x) \rmd x\right]\backasymVar[\gamma]{s+1}{T}{h^2_{s+1}}\eqsp,
\end{multline*}
with $h^1_{s+1}(x)\eqdef h(x)\backDist[][\gamma]{s+1}{T}[q(x,\cdot)\gamma_{s+1}^{-1}]$ and $h^2_{s+1}(x)\eqdef \gamma_{s+1}^{-1}(x)\filt{\Xinit,s}[h(\cdot)q(\cdot,x)]$. In the case where $\kiss[smooth]{s}{T}(x_s,x_{s+1}; x_s) = \kissforward{s}{s}(x_{s-1}, x_s)$ in \eqref{eq:inst-fearnhead:linear} and $\adjfunc[smooth]{s}{T}{x,x'} = \adjfuncforward{s}(x)\adjfunc{s}{T}{x'}$, the smoothing distribution approximation given by the $\fwt$ algorithm is obtained by reweighting the particles obtained in the forward filtering pass and $\asymVarFearnhead[\Xinit]{s}{T}{h}$ may be compared to $\asymVarDoucet[\Xinit]{\forward}{s}{T}{h}$ as both approximations of $\post[\Xinit]{s}{T}[h]$ are based on the same particles (associated with different importance weights). In this case, the two last terms in \eqref{eqdef:asymvarfearnhead} are easily interpreted in the case  $\adjfunc{s}{T}{} = \gamma_{s+1}$:
\[
\filt{\Xinit,s-1} \otimes \backDist[][\gamma]{s+1}{T} \left[\adjfunc[smooth]{s}{T}{} \odot \fakeprior^{-1}_{s+1} \right] = \filt{\Xinit,s-1}[ \adjfuncforward{s}]\backDist[][\gamma]{s+1}{T}[\adjfunc{s}{T}{}\fakeprior^{-1}_{s+1}] = \filt{\Xinit,s-1}[ \adjfuncforward{s}]
\]
and by Jensen's inequality,
\begin{align*}
&\filt{\Xinit,s-1} \otimes \backDist[][\gamma]{s+1}{T}\left[ \int \smwghtfunc{s}{T}(\cdot;x) q^{[2]}(\cdot,x) g_s(x) h^2(x) \rmd x \odot \fakeprior^{-1}_{s+1} \right] \\
&\hspace{.5cm}= \int \filt{\Xinit,s-1}(\rmd x_{s-1})\omega_s(x_{s-1},x)g_s(x)q(x_{s-1},x)\backDist[][\gamma]{s+1}{T}[q^2(x,\cdot)\gamma_{s+1}^{-2}]h^2(x) \rmd x\eqsp,\\
&\hspace{.5cm}\ge \int \filt{\Xinit,s-1}(\rmd x_{s-1})\omega_s(x_{s-1},x)g_s(x)q(x_{s-1},x)(h^1_{s+1}(x))^2 \rmd x\eqsp.
\end{align*}
Therefore, by Proposition~\ref{prop:CLT-backward} and Theorem~\ref{thm:CLT-Doucet-Forward-Backward-Smoother}
\[
\asymVarFearnhead[\Xinit]{s}{T}{h} \ge \frac{\asymVar[\Xinit]{s}{h^1_{s+1}} + \backasymVar[\gamma]{s+1}{T}{h^2_{s+1}}}{\left(\filt{\Xinit,s} \otimes \backDist[][\gamma]{s+1}{T} \left[q\odot \fakeprior_{s+1}^{-1} \right]\right)^2}=\asymVarDoucet[\Xinit]{\forward}{s}{T}{h}\eqsp,
\]
where the last inequality comes from Theorem~\ref{thm:CLT-Two-Populations}. The same inequality holds for $\asymVarDoucet[\Xinit]{\backward}{s}{T}{h}$ when $\kiss[smooth]{s}{T}(x_{s-1},x_{s+1}; x_s) = \kiss{s}{T}(x_{s+1}, x_s)$ in \eqref{eq:inst-fearnhead:linear}.


\begin{rem}
Under the strong mixing assumptions H\ref{assum:mix} and H\ref{assum:mix:gamma}, time uniform bounds for the asymptotic variances of  the two-filter approximations of the marginal smoothing distributions may be obtained.
\begin{enumerate}[(i)]
\item If A\ref{assum:bound-likelihood} and A\ref{assum:borne-FFBS} hold uniformly in $T$ and if H\ref{assum:mix} holds, then it is proved in \cite{douc:garivier:moulines:olsson:2011} that there exists $C>0$ such that for all $s \geq 0$ and all $h \in \functionset[b]{X}$, the asymptotic variance $\asymVar[\Xinit]{s}{h}$ defined in Proposition~\ref{prop:CLT-forward} satisfies:
\[
\asymVar[\Xinit]{s}{h} \le C \esssup{h}^2\eqsp.
\]
\item Following the same steps, if A\ref{assum:bound-likelihood} and A\ref{assum:borne-TwoFilters} hold uniformly in $T$  and if H\ref{assum:mix} and H\ref{assum:mix:gamma} hold, there exists $C>0$ such that for all $0\le t\le T$ and all $h \in \functionset[b]{X}$, the asymptotic variance $\backasymVar{t}{T}{h}$ defined in Proposition~\ref{prop:CLT-backward} satisfies:
\[
\backasymVar[\gamma]{t}{T}{h}\le C \esssup{h}^2\eqsp.
\]
\item As a consequence, if A\ref{assum:bound-likelihood}, A\ref{assum:borne-FFBS} and A\ref{assum:borne-TwoFilters} hold uniformly in $T$  and  if H\ref{assum:mix} and H\ref{assum:mix:gamma} hold, the asymptotic variances $\asymVarJoint{s,t}{T}{h}$, $\asymVarDoucet[\Xinit]{\forward}{s}{T}{h}$, $\asymVarDoucet[\Xinit]{\backward}{s}{T}{h}$ and $\asymVarFearnhead[\Xinit]{s}{T}{h}$ defined in Theorem~\ref{thm:CLT-Two-Populations}, Theorem~\ref{thm:CLT-Fearnhead-smoother}  and Theorem~\ref{thm:CLT-Doucet-Forward-Backward-Smoother} are all uniformly bounded.
\end{enumerate}
\end{rem}
Note that \cite{persing:jasra:2013} proposed an estimator of the likelihood of the observations using a two-filter decomposition. This particle based estimator is unbiased and the authors established a CLT, see \cite[Theorem~2.1]{persing:jasra:2013}. This CLT is the sum of two contributions associated with the forward and backward particle filters as the CLT we establish for the marginal smoothing distributions. However, due to the complex dependency of the asymptotic variance on $\{\gamma_t\}_{t\ge 0}$, there is no general intuition in this result either to choose this sequence sharply in a generic framework.